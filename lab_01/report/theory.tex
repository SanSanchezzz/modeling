%\chapter{Теоретические сведения}

\textbf{Цель работы:} Изучить методы решения задачи Коши для ОДУ,
применив приближенный аналитический метод Пикара и численный метод
Эйлера в явном и неявном вариантах.

\textbf{Задание:} Решить уравнение (формула \ref{eq:task}),
не имеющее аналитического решения.

\begin{equation}\label{eq:task}
    \begin{cases}
        u'(x) = f(x,u) \\
        u(\xi) = y
    \end{cases}
\end{equation}

Уравнение можно решить \textbf{методом Пикара} (формула \ref{eq:pik})

\begin{equation}\label{eq:pik}
    \begin{matrix}
        y^{(s)}(x) = \eta + \int_0^x f(t, y^{(s-1)}(t)) dt \\
        \\
        y^{(0)} = \eta \\
    \end{matrix}
\end{equation}

Для задачи получим 4 приближения (формулы \ref{eq:p1}, \ref{eq:p2},
\ref{eq:p3}, \ref{eq:p4})

\begin{equation}\label{eq:p1}
    y^{(1)} = \frac{x^3}{3}
\end{equation}

\begin{equation}\label{eq:p2}
    y^{(2)} = \frac{x^3}{3} + \frac{x^7}{21}
\end{equation}

\begin{equation}\label{eq:p3}
    y^{(3)} = \frac{x^3}{3} + \frac{x^7}{21} + \frac{2 \cdot x^8}{2079}
    + \frac{x^{15}}{59535}
\end{equation}

\begin{multline}\label{eq:p4}
        y^{(4)} = \frac{x^3}{3} + \frac{x^7}{21} +
        \frac{2 \cdot x^8}{2079} +
        \frac{x^{15}}{59535} +
        \frac{2 \cdot x^{15}}{93555} +
        \frac{2 \cdot x^{19}}{3393495} + \\
        \frac{2 \cdot x^{29}}{2488563} +
        \frac{2 \cdot x^{23}}{86266215} +
        \frac{x^{23}}{99411543} +
        \frac{2 \cdot x^{27}}{3341878155} +
        \frac{x^{31}}{109876902975}
\end{multline}

Также уравнение решается \textbf{численным методом Эйлера}

Явная схема (формула \ref{eq:ex})

\begin{equation}\label{eq:ex}
    y_{n+1} = y_n + h \cdot f(x_n, y_n)
\end{equation}

Неявная схема (формула \ref{eq:im})

\begin{equation}\label{eq:im}
    y_{n+1} = y_n + h \cdot (f(x_{n+1}, y_{n+1}))
\end{equation}

Ниже приведены листинги реализованных методов:

\begin{lstlisting}[caption=Метод Пикара]
def methodPicara(valuesX, y_0):
    def y_1(x):
        return pow(x, 3) / 3.0

    def y_2(x):
        return y_1(x) + pow(x, 7) / 63.0

    def y_3(x):
        return  y_2(x) + 2 * pow(x, 11) / 2079.0 + pow(x, 15) / 59535.0

    def y_4(x):
        return y_3(x) + 2 * pow(x, 15) / 93555.0 + 82 * pow(x, 19) / 37328445.0 + 662 * pow(x, 23) / 10438212015.0 + \
                4 * pow(x, 27) / 3341878155.0 + pow(x, 31) / 109876902975.0

    result = [[y_0, y_0]]

    for i in range(1, len(valuesX)):
        result.append([y_3(valuesX[i]), y_4(valuesX[i])])

    return result
\end{lstlisting}

\begin{lstlisting}[caption=Явная схема метода Эйлера]
def methodEulerExplicit(valuesX, step, y):
    result = []

    for i in range(len(valuesX)):
        if (y == float('inf')):
            break
        y += step * (y * y + valuesX[i] * valuesX[i])
        result.append(y)

    return result
\end{lstlisting}

\begin{lstlisting}[caption=Неявная схема метода Эйлера]
def methodEulerNExplicit(valuesX, step, y):
    result = []

    for i in range(len(valuesX)):
        discr = 1 - step * 4 * (y + step * (valuesX[i] * valuesX[i]))

        if (discr < 0):
            break

        y = (1 - sqrt(discr)) / (2 * step)
        result.append(y)

    return result
\end{lstlisting}

